\section{Conclusion}

Pour conclure, nous avons pu étudier les différentes méthodes de compression sans perte. Nous avons constaté que chaque méthode a ses avantages et ses inconvénients. Ainsi, il est crucial de choisir judicieusement la méthode de compression en fonction des données à compresser. Par exemple, si les données contiennent des motifs répétitifs, les méthodes à dictionnaire seraient un meilleur choix. Si au contraire les motifs sont aléatoires, nous choisirons une méthode statistique. Il faut aussi donner une attention particulière aux paramètres spécifiques de chaque méthodes telle que la taille du dictionnaire ou le choix du codage pour les méthodes statistiques. Il est donc important de bien comprendre le fonctionnement de chaque méthode de compression sans perte pour une sélection optimale de la méthode et de ses paramètres.