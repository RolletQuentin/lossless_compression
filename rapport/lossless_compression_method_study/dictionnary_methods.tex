\subsection{Méthodes à dictionnaire}

    \subsubsection{Comparaion LZ78/LZW}

    La différence entre LZ78 et LZW est la manière dont les dictionnaires sont gérés. Dans LZ78, le dictionnaire est géré par un arbre de préfixes. Dans LZW, le dictionnaire est géré par une table de hachage.

    \medskip

    En termes de performances, LZ78 surpasse LZW. La vitesse de LZ78 est accrue du fait que la recherche de mots dans un arbre de préfixes est plus efficace que dans une table de hachage, tandis que sa consommation de mémoire est moindre puisqu'il stocke les mots dans un arbre de préfixes plutôt que dans une table de hachage, contrairement à LZW.

    \subsubsection{Meilleures configurations}

    L'algorithme LZ78 fonctionne bien sur des données contenant des motifs répétitifs. En effet, plus il y a de motifs répétitifs, plus le dictionnaire sera rempli et plus la compression sera efficace. Par exemple, si on compresse un fichier contenant des motifs répétitifs, on obtiendra un taux de compression élevé. En revanche, si on compresse un fichier contenant des motifs aléatoires, on obtiendra un taux de compression faible. Il est aussi bénéfique d'utiliser un dictionnaire assez grand pour stocker un grand nombre de séquences de caractères, mais il faut faire attention de ne pas utiliser un dictionnaire trop grand pour éviter de gaspiller de la mémoire.

    Pour LZW, c'est un peu plus compliqué car la taille du dictionnaire est fixe et déterminé à l'avance. Généralement nous utilisons un dictionnaire à 256 entrées correspondant à la table ASCII, mais certaines implémentations de l'algorithmes permettent de changer la taille du dictionnaire d'entrée voire même de changer le dictionnaire de façon dynamique. On peut donc essayer de rentrer plusieurs tailles de dictionnaires différentes ou tester des stratégies différentes d'adaptation dynamique pour déterminer la configuration optimale dans chaque cas. De la même manière que pour LZ78, on obtient une meilleure compression si le fichier contient des motifs répétitifs.