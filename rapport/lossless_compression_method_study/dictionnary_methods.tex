\subsection{Méthodes à dictionnaire}

    \subsubsection{Comparaion LZ78/LZW}

    La différence entre LZ78 et LZW est la manière dont les dictionnaires sont gérés. Dans LZ78, le dictionnaire est géré par un arbre de préfixes. Dans LZW, le dictionnaire est géré par une table de hachage.

    \medskip

    En terme de performance, LZ78 est plus performant que LZW. En effet, LZ78 est plus rapide que LZW car la recherche d'un mot dans un arbre de préfixes est plus rapide que la recherche d'un mot dans une table de hachage. De plus, LZ78 est plus performant que LZW car LZ78 utilise moins de mémoire que LZW. En effet, LZ78 utilise un arbre de préfixes pour stocker les mots alors que LZW utilise une table de hachage pour stocker les mots.

    \subsubsection{Meilleures configurations}

    L'algorithme LZ78 fonctionne bien sur des données contenant des motifs répétitifs. En effet, plus il y a de motifs répétitifs, plus le dictionnaire sera rempli et plus la compression sera efficace. Par exemple, si on compresse un fichier contenant des motifs répétitifs, on obtiendra un taux de compression élevé. En revanche, si on compresse un fichier contenant des motifs aléatoires, on obtiendra un taux de compression faible. Il est aussi bénéfique d'utiliser un dictionnaire assez grand pour stocker iun grand nombre de séquences de caractères, mais il faut faire attention de ne pas utiliser un dictionnaire trop grand pour éviter de gaspiller de la mémoire.

    Pour LZW, c'est un petit peu plus compliqué car la taille du dictionnaire est fixe et déterminé à l'avance. Généralement nous utilisons un dictionnaire à 256 entrées correspondant à la table ASCII, mais certaines implémentations de l'algorithmes permettent de changer la taille du dictionnaire d'entrée voire même de changer le dictionnaire de façon dynamique. On peut donc essayer de rentrer plusieurs tailles de dictionnaires différentes ou tester des stratégies différentes d'adaptation dynamique pour déterminer la configuration optimale dans chaque cas.