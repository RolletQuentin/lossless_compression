\section{Introduction}

Les algorithmes de compression, véritables énigmes mathématiques complexes, revêtent une importance cruciale dans le domaine de l'informatique. Alors que certains optent pour des méthodes de compression avec perte, particulièrement adaptées à des applications telles que la compression d'images, cette approche n'est pas toujours idéale, notamment lorsque la préservation intégrale des données est primordiale.

\bigskip

Dans ce rapport, nous explorerons minutieusement les mécanismes des algorithmes de compression sans perte. Ces derniers se distinguent par leur capacité à réduire la taille des données sans altérer leur contenu, une caractéristique essentielle dans de nombreux contextes. Nous aborderons les différentes phases du processus, de la conception à l'implémentation, en passant par une évaluation approfondie des méthodes existantes.

\bigskip

La première section de notre étude se penchera sur la conception des algorithmes, mettant en lumière la répartition des tâches et la mise en place rigoureuse des tests, deux aspects cruciaux dans le développement de solutions efficaces. Par la suite, nous plongerons dans l'implémentation, examinant en détail la méthode Run-Length Encoding (RLE), les approches statistiques, et les méthodes à dictionnaire.

\medskip

La section suivante approfondira notre analyse en étudiant de près les différentes méthodes de compression, notamment le RLE, les méthodes statistiques avec une comparaison approfondie entre Huffman et Shannon-Fano, ainsi que les méthodes à dictionnaire, avec un examen détaillé des différences entre LZ78 et LZW.\@ Nous conclurons cette section en identifiant les configurations les plus performantes au sein de chaque méthode.

\medskip

Enfin, cette étude se clôturera par une conclusion éclairante, résumant les principales conclusions tirées de notre exploration approfondie des algorithmes de compression sans perte. À travers ce rapport, nous chercherons à offrir une vision complète et nuancée des différentes approches, permettant ainsi de guider les choix futurs dans le domaine de la compression des données.